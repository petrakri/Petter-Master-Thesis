\chapter{Introduction}


\section{Objective of thesis}
The goal of this thesis is to create a measurement setup like \textit{Skiselector} or similar setups, using cheaper commercial type sensors which can match the results on span curves with an addition of generating pressure distribution plots.

\section{Research questions}
\label{sec:researchquestions}
As a conclusion of the research done for this master thesis, the following research questions arose:

\begin{enumerate}[label=\textbf{\Alph*}]
    \item How can we identify twisting in cross-country skis and how does this affect the performance of a skier?
    \item What relation exists between the ski's mechanical properties and the friction coefficient between the running phase of the ski and snow?
    \item How can we classify the performance of cross-country skis based on the mechanical properties?
\end{enumerate}

\section{Key results}
\begin{itemize}
    \item \textbf{Written at the end of the thesis}
\end{itemize}

\section{Research Context}
\label{sec:reasearchcontext}
General research in this field has been focused on finding skis with matching mechanical properties and the preparation and treatment of the ski sole.
In the process of prototyping a measurement setup to measure pressure distribution, several sensors need to be taken into consideration. 

Certain characteristics or mechanical properties of a cross-country ski can be extracted by the measurements from a measurement bench equipped with several sensors placed along the ski. This can be done by registering forces working on each sensor when applying force down on to a central point on the ski. This can further be used to find a pair of matching skis. Matching skis mean, a pair of skis that have similar mechanical properties which contribute to the performance of the skier.

\section{Thesis outline}
\textbf{Chapter 2} explains the concept of cross-country skiing and discusses the important mechanical properties that needs to be considered. These mechanical properties are key features that contributes to the skiers overall performance. \newline
\textbf{Chapter 3} consider components and electrical circuits required to collect data from skis. This is the basis for creating a measurement device to collect data from cross-country skis. At the end of the chapters, calibration of the sensors and the measurement system is discussed. \newline
\textbf{Chapter 4} considers the options of microcontrollers, which is needed to read and convert the measurements to sampled data. \newline
\textbf{Chapter 5} explains different software used to analyze data and designing the electric circuits used in the thesis. \newline
\textbf{Chapter 6} considers the measured data and shows the key results from the measured data. This is followed by \textbf{Chapter 7} which discusses these results further and concludes the research done.
