\chapter{Conclusion}
The aspect of pairing two skis with equal characteristics was considering in \citep{backstrom_essential_2008}, where the Swedish national team had been using a ski measurement device to measure the weight distribution for three years. As a result of these measurements, they had been able to measure all their skis and save the data for further matching of skis to the athletes. This is also an aspect that is interesting for this Master Thesis.

As a product of Felix Breitschädels research in 2013, a new test bench was developed to measure the chamber height and the span curves in individual skis to create a profile of the measured ski. This went under the category of a Finite Element Simulation to produce heat maps for the weight distribution. One of the problems discussed in the publication was the sensitivity of the model. The FE Simulation had the capability of producing good estimates of the span curve with given applied force $N$. The sensitivity mentioned was due to the inaccuracies in the measured deflection and the double derivative of the formula for calculating the bending stiffness:
\begin{equation}
    El = \frac{M}{k} , k \approx \frac{\partial^2 u}{\partial x^2} 
\end{equation}

\section{Further work}
What can be done in the future