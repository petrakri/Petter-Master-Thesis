\chapter{Discussion}
This chapter explains the results and analyses them.


\section{Mechanical properties}
How the mechanical properties can affect gliding speeds

\section{Pressure distribution}
What the pressure distribution means

\section{Span curve}
What the span curve means

\section{Friction affected by mechanical properties}
See if I can relate friction to the mechanical properties

\section{How to choose a ski}
\label{sec:choosingaski}
Looking into how we should choose a ski, we investigate the mechanical properties and how they relate to snow and weather conditions.
Choosing the correct ski for the ideal snow condition is based on the stiffness of the ski. For Classic skis, the choice of stiffness, $k$, to press the chamber height down to $0.2mm$ was around 66\% and 67\% of the body weight for men and women respectively and for warm conditions an increase to 77\% for both men and women.

Adjusting factors such as gliding wax and mechanical properties of the ski for different weather conditions and temperatures can result in better gliding speeds. In addition to these adjustable factors, a cold and warm condition is defined for each pair of skis. For temperatures below -3\textdegree C, skis for cold conditions are chosen. The skis for warm conditions typically are stiffer \citep{breitschadel_variation_2012}. We choose the ski with regard to stiffness to weight ratio for a given weather condition.

The aspect of pairing two skis with equal characteristics was considering in \citep{backstrom_essential_2008}, where the Swedish national team had been using a ski measurement device to measure the weight distribution for three years. As a result of these measurements, they had been able to measure all their skis and save the data for further matching of skis to the athletes